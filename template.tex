\documentclass[a4paper,11pt,pdftex,english,norsk]{article}

\usepackage{listings}
\usepackage{babel}
\usepackage{amsmath}
\usepackage{graphicx}
\usepackage[utf8]{inputenc}
\usepackage{epsfig}
\usepackage{graphics}
\usepackage{palatino}
\renewcommand{\ttdefault}{lmtt}
\usepackage{enumerate}
\usepackage{mdwlist}
\usepackage{geometry}
%\usepackage{textcomp}
%\usepackage{type1cm}
\usepackage[table]{xcolor}
\usepackage{varioref}
\usepackage{url}
\usepackage[bookmarks=true, linkcolor=blue,
citecolor=blue,urlcolor=blue,colorli nks=true, breaklinks=true,
pagebackref=true, hyperindex=true,bookmarksopen=true]{hyperref}
\usepackage{enumitem}
\newlist{arrowlist}{itemize}{1}
\setlist[arrowlist]{label=$\Rightarrow$}


\frenchspacing

%% Block style paragrphs
\setlength\parskip{\medskipamount}
\setlength\parindent{0pt}

\makeatletter
\renewcommand{\topfraction}{.9}
\renewcommand{\bottomfraction}{.8}
\renewcommand{\textfraction}{.15}
\renewcommand{\floatpagefraction}{.66}
\renewcommand{\dbltopfraction}{.66}
\renewcommand{\dblfloatpagefraction}{.66}
\setcounter{topnumber}{9}
\setcounter{bottomnumber}{9}
\setcounter{totalnumber}{20}
\setcounter{dbltopnumber}{9}

\makeatother



\title{Project Report Template}
\author{987654, 987651 and 987621}

\begin{document}

\maketitle

\abstract{From Landes~\cite{Landes:66}: The abstract is of utmost
  importance, for it is read by 10 to 500 times more people than hear or
  read the entire article. It should not be a mere recital of the subjects
  covered. Expressions such as "is discussed" and "is described" should
  never be included! The abstract should be a condensation and
  concentration of the essential information in the paper.}

\thispagestyle{empty}

\clearpage
\pagenumbering{roman}
\setcounter{page}{1}
\tableofcontents

\clearpage
\pagenumbering{arabic}

\section{Introduction}

This is a short template you can use for your project report. Feel free to
make your own modifications to the structure.

Join together to form project groups of max 3 people in each group (you can
also be just 1 if you prefer to work alone). Your teacher will list project
proposals for you. Here are the project titles from 2007-2010:
\begin{itemize}
\item Amanda Network Backup
\item Browser configuration
\item IDS
\item OpenVPN
\item F-Secure
\item Automatiske oppdateringer
\item Controlled Malware Infection
\item WSUS - Windows Server Update Service
\item Microsoft Offline Files
\item SNORT implementation as IDS and IPS on Windows and Linux
\item Konvertering av lab A115 fra Cfengine 2 til Cfengine 3
\item Virtualisering – VMware vs. KVM
\item Trac
\item OpenVPN and PortKnocking
\item System Center Configuration Manager
\item Cfengine på Windows
\item Remote OS and file deployment
\item Host-based IDS: AppArmour and SELinux
\item Samba 3.2 as a replacement for Active Directory
\item NFSv4
\item Automatic Installations: Red Hat Kickstart using Fedora
\item Honeynet
\item Automatisk installering med Ghost
\item Automatisk installering med RIS
\end{itemize}
For the topic you choose write a report og 5-15 pages plus appendices (and prepare a 15
minute presentation).

Good resources might be:
\begin{itemize}
\item Dag Langmyrs \LaTeX book~\cite{Langmyr:03}
\item Journal articles like~\cite{Klein:09}
\item Conference proceedings articles like~\cite{Begnum:07} 
\item Wikipedia pages like~\cite{wikipedia:kerberos:10}
\end{itemize}

NOTE: it is EXTREMELY important that you write in your own words, and not
just translate something you find. 

If you want to write ``the perfect introduction'' I strongly suggest you
read Claerbout's ``Scrutiny of the introduction''~\cite{Claerbout:95}.

\section{SLA/RECSPEC}

Please state why you are doing this project. Who will benefit from it? What
can the users expect in terms of stability and performance? What can be
measured so this can be a real ``Service Level Agreement''? Think carefully
about this in the beginning of the project, imagine you are doing this to
make money. To send an invoice (faktura) after project completion you have
to explain to the customer that the project has now been completed
according to what we agreed on when the project began. In other words, make
sure you are not ``just doing this for fun''.

\section{Survey of similar projects}

There are always someone who has done what you are going to do, or
something related. Breifly describe at least two or three similar projects,
to motivate what your contribution will be.

\section{Description of your experiment}

Describe what you are going to do and why, relate it the the previous work
you described in the previous chapter.

\section{Results and discussion}

This is the longest chapter, feel free to include some figures. Include
discussion of all problems you have encountered.

\section{Security aspects}

This does not have to be a separate part, but many times security concerns
are what is stopping us.

\section{Conclusions}

So what is the punchline, does it work or not? Do you recommend doing this
for others?

\bibliographystyle{acmdoi}
\bibliography{template}

\end{document}
