\documentclass[a4paper,11pt,pdftex,english,norsk]{article}

\usepackage{listings}
\usepackage{babel}
\usepackage{amsmath}
\usepackage{graphicx}
\usepackage[utf8]{inputenc}
\usepackage{epsfig}
\usepackage{graphics}
\usepackage{palatino}
\renewcommand{\ttdefault}{lmtt}
\usepackage{enumerate}
\usepackage{mdwlist}
\usepackage{geometry}
%\usepackage{textcomp}
%\usepackage{type1cm}
\usepackage[table]{xcolor}
\usepackage{varioref}
\usepackage{url}
\usepackage[bookmarks=true, linkcolor=blue,
citecolor=blue,urlcolor=blue,colorli nks=true, breaklinks=true,
pagebackref=true, hyperindex=true,bookmarksopen=true]{hyperref}
\usepackage{enumitem}
\newlist{arrowlist}{itemize}{1}
\setlist[arrowlist]{label=$\Rightarrow$}


\frenchspacing

%% Block style paragrphs
\setlength\parskip{\medskipamount}
\setlength\parindent{0pt}

\makeatletter
\renewcommand{\topfraction}{.9}
\renewcommand{\bottomfraction}{.8}
\renewcommand{\textfraction}{.15}
\renewcommand{\floatpagefraction}{.66}
\renewcommand{\dbltopfraction}{.66}
\renewcommand{\dblfloatpagefraction}{.66}
\setcounter{topnumber}{9}
\setcounter{bottomnumber}{9}
\setcounter{totalnumber}{20}
\setcounter{dbltopnumber}{9}

\makeatother



\title{GlusterFS: Changing bricks without downtime or data loss}
\author{140139}

\begin{document}

\maketitle

\abstract{SLA}

\thispagestyle{empty}

\clearpage
\pagenumbering{roman}
\setcounter{page}{1}
\tableofcontents

\clearpage
\pagenumbering{arabic}

\section{Introduction}


\section{SLA/RECSPEC}

By doing this project I wish to learn as much as possible about GlusterFS and file systems in general. I see this as an opportunity to acquire practical knowledge, in a real scenario that otherwise would be very hard to get. Also, besides helping me to develop myself as an capable system administrator, this project will have direct positive effect on the university infrastructure, by expanding GlusterFS effective storage space and proving its scalabillity.

To accomplish this, I will start by simulating the real scenario on virtual machines and learning how GlusterFS works. I intend to repeat this process as many times as needed to get a very good understanding of the brick substitution process.


\section{Survey of similar projects}


\section{Description of your experiment}


\section{Results and discussion}


\section{Security aspects}


\section{Conclusions}


\bibliographystyle{acmdoi}
\bibliography{template}

\end{document}
